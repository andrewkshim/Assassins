%%%%%%%%%%%%%%%%%%%%%%%%%%%%%%%%%%%%%%%%%
% Structured General Purpose Assignment
% LaTeX Template
%
% This template has been downloaded from:
% http://www.latextemplates.com
%
% Original author:
% Ted Pavlic (http://www.tedpavlic.com)
%
% Note:
% The \lipsum[#] commands throughout this template generate dummy text
% to fill the template out. These commands should all be removed when 
% writing assignment content.
%
%%%%%%%%%%%%%%%%%%%%%%%%%%%%%%%%%%%%%%%%%

%----------------------------------------------------------------------------------------
%	PACKAGES AND OTHER DOCUMENT CONFIGURATIONS
%----------------------------------------------------------------------------------------

\documentclass{article}

\usepackage{fancyhdr} % Required for custom headers
\usepackage{lastpage} % Required to determine the last page for the footer
\usepackage{extramarks} % Required for headers and footers
\usepackage{graphicx} % Required to insert images
\usepackage{lipsum} % Used for inserting dummy 'Lorem ipsum' text into the template
\usepackage[colorlinks=true, urlcolor=blue]{hyperref}

% Margins
\topmargin=-0.45in
\evensidemargin=0in
\oddsidemargin=0in
\textwidth=6.5in
\textheight=9.0in
\headsep=0.25in 

\linespread{1.1} % Line spacing

% Set up the header and footer
\pagestyle{fancy}
\lhead{\hmwkAuthorName} % Top left header
\chead{\hmwkClass\ \hmwkTitle} % Top center header
\rhead{\dueDate} % Top right header
\lfoot{\lastxmark} % Bottom left footer
\cfoot{} % Bottom center footer
\rfoot{Page\ \thepage\ of\ \pageref{LastPage}} % Bottom right footer
\renewcommand\headrulewidth{0.4pt} % Size of the header rule
\renewcommand\footrulewidth{0.4pt} % Size of the footer rule

\setlength\parindent{0pt} % Removes all indentation from paragraphs

%----------------------------------------------------------------------------------------
%	DOCUMENT STRUCTURE COMMANDS
%	Skip this unless you know what you're doing
%----------------------------------------------------------------------------------------

% Header and footer for when a page split occurs within a problem environment
\newcommand{\enterProblemHeader}[1]{
\nobreak\extramarks{#1}{#1 continued on next page\ldots}\nobreak
\nobreak\extramarks{#1 (continued)}{#1 continued on next page\ldots}\nobreak
}

% Header and footer for when a page split occurs between problem environments
\newcommand{\exitProblemHeader}[1]{
\nobreak\extramarks{#1 (continued)}{#1 continued on next page\ldots}\nobreak
\nobreak\extramarks{#1}{}\nobreak
}

\setcounter{secnumdepth}{0} % Removes default section numbers
\newcounter{homeworkProblemCounter} % Creates a counter to keep track of the number of problems

\newcommand{\homeworkProblemName}{}
\newenvironment{homeworkProblem}[1][Problem \arabic{homeworkProblemCounter}]{ % Makes a new environment called homeworkProblem which takes 1 argument (custom name) but the default is "Problem #"
\stepcounter{homeworkProblemCounter} % Increase counter for number of problems
\renewcommand{\homeworkProblemName}{#1} % Assign \homeworkProblemName the name of the problem
\section{\homeworkProblemName} % Make a section in the document with the custom problem count
\enterProblemHeader{\homeworkProblemName} % Header and footer within the environment
}{
\exitProblemHeader{\homeworkProblemName} % Header and footer after the environment
}

\newcommand{\problemAnswer}[1]{ % Defines the problem answer command with the content as the only argument
\noindent\framebox[\columnwidth][c]{\begin{minipage}{0.98\columnwidth}#1\end{minipage}} % Makes the box around the problem answer and puts the content inside
}

\newcommand{\homeworkSectionName}{}
\newenvironment{homeworkSection}[1]{ % New environment for sections within homework problems, takes 1 argument - the name of the section
\renewcommand{\homeworkSectionName}{#1} % Assign \homeworkSectionName to the name of the section from the environment argument
\subsection{\homeworkSectionName} % Make a subsection with the custom name of the subsection
\enterProblemHeader{\homeworkProblemName\ [\homeworkSectionName]} % Header and footer within the environment
}{
\enterProblemHeader{\homeworkProblemName} % Header and footer after the environment
}

\newcommand\textline[4][t]{%
  \par\smallskip\noindent\parbox[#1]{.200\textwidth}{\raggedright#2}%
  \parbox[#1]{.600\textwidth}{\centering\Large\textbf{#3}}%
  \parbox[#1]{.200\textwidth}{\raggedleft#4}\par\smallskip%
  \line(1,0){470}
}
   
%----------------------------------------------------------------------------------------
%	NAME AND CLASS SECTION
%----------------------------------------------------------------------------------------

\newcommand{\hmwkTitle}{Initial Project Proposal} % Assignment title
\newcommand{\hmwkDueDate}{Monday,\ January\ 1,\ 2012} % Due date
\newcommand{\hmwkClass}{CS\ 344} % Course/class
\newcommand{\hmwkClassTime}{10:30am} % Class/lecture time
\newcommand{\hmwkClassInstructor}{Jones} % Teacher/lecturer
\newcommand{\hmwkAuthorName}{Andrew Shim} % Your name
\newcommand{\dueDate}{17 Oct 2013}

\begin{document}\thispagestyle{empty}

%----------------------------------------------------------------------------------------
%	TITLE 
%----------------------------------------------------------------------------------------

\textline[t]{\hmwkAuthorName}{\hmwkClass\ \hmwkTitle}{\dueDate}

%----------------------------------------------------------------------------------------

\section{Description of Final Product}
I want to make an augmented reality (AR) game for the iPhone. Here are some samples
of AR games for mobile devices:
\begin{itemize}
  \item\href{http://www.youtube.com/watch?v=rB5xUStsUs4}{AR Defender}
  \item\href{http://www.youtube.com/watch?v=Ce1U_9DRcac}{Zombies Everywhere!}
\end{itemize}
My game is going to be an AR adaptation of the game 
\href{http://en.wikipedia.org/wiki/Assassin_(game)}{Assassins} and
will go by the same name for now (I know its no ``Peckerman'' or 
``Sil y Vega Peck''). The basic premise of \texttt{Assassins} is similar to 
\href{http://en.wikipedia.org/wiki/Tag_(game)}{Tag}
but with a couple of ``killer'' modifications; you gather a group
of players then the game host/referee will assign each player a target
(another player) to ``assassinate''. Assassination can entail a variety of
actions, some examples I have seen are: poking the person in the back with a spoon 
(basically simulating that you're stabbing them, I know it is violent but the 
game's intentions are completely harmless I swear), 
touching them with a (clean) sock, and shooting them with a 
\href{http://www.hasbro.com/nerf/en_us/}{Nerf Gun}. Over time, several 
embellishments to the game have been added. For example, a common practice
is to allow players to carry a safety item (usually something 
cumbersome or embarrassing such as a pair of underwear) that gives them
immunity to attacks as long as the item is visible.
\\\\
My current vision for the game is that users will use their camera 
viewfinders in order to claim their targets (more details below). 
The attack will not be instantaneous, instead the target will receive
a notification about the attack and will have the opportunity 
to evade it if they notice in time. The attackers' cameras will be able to
recognize their targets because every target will be required to 
wear a brightly colored square (they can print these out themselves)
that must be clearly visible. If the square is not visible, the attackers
can attack by taking a photo of their targets and sending it to the game host, who can 
then confirm the attack and eliminate the target. The safety item will also
be a part of the game and will be implemented with brightly colored 
squares as well (players will need to get used to wearing squares).
If the attacker's camera sees the safety item on the target, 
the attacker will not be able to attack.
\\\\
This is the very basic functionality I want in the game. Time permitting,
I plan on adding features such as traps and powerups. An example of a trap
is a landmine, which a player might place in a highly trafficked area
to blow up unsuspecting players. The landmines would use GPS
coordinates, so if a player was within some trigger distance to a landmine they would
get eliminated. A powerup example is a one-time use shield, allowing a player to survive
an attempt on their virtual life.

\section{Using Computer Graphics}
One cannot have AR without computer graphics (CG). \texttt{Assassins} will use various computer 
graphics to enhance the player experience. 
\subsection{CG in the Attacker's View}
As attackers are viewing their targets through their phone cameras, if the target is wearing the
safety item (the colored square) I will render an aura of light around the target,
which will be done through a particle generator.
Also, I will to use computer vision (\href{http://opencv.org/}{OpenCV}) to detect the 
targets and form a bounding box around them. 
Once attackers have their target on their screens, they will need to tap 
on the target and ``shoot'' them down.
A tap will send out an energy ball that will get
smaller as it goes farther into the negative z direction. Once the 
sphere is small enough and if it is still within the target's bounding box,
the target will have been hit.
\subsection{CG in Any Player's View}
If I get to implement powerups, I will make them such that
powerups appear at random GPS coordinates. Note that I plan on 
having the referee bound the game map to a square defined by four GPS 
coordinates so if a group of Duke students are playing, a powerup will
not pop up in Venice. Players will be notified if they are near a powerup, after
which they will need to pull out their phone cameras. If they are in the correct
location, their camera will show a visible sparkle (more particle generators) 
in the center of the screen
and they will pick up the powerup, after which the powerup will no longer be 
available to other players.
\subsection{Related CG Topics}
The CG enhancements fall under a range of topics we have covered in class. Using the
energy balls as the primary example,
I will need to set up the \textbf{material properties} to make them look like 
\href{http://blog.tricyclestudios.com/?tag=hadouken-photo}{this} (top photo).
If you are reading this as a print out then you probably cannot click the 
link, so just imagine the energy balls as shiny blue balls of light.
Also, as the energy balls travel in the scene, I will make it such that 
they emit light (\textbf{emissive}), and to make it realistic I will modify
the scene's \textbf{shading and lighting} based on the energy ball's location.
To make them look even cooler, I will use a \textbf{particle generator} to give
them an aura. I plan on using a combination of \textbf{OpenGL} and iOS 7's new 
\href{https://developer.apple.com/library/ios/documentation/GraphicsAnimation/Conceptual/SpriteKit_PG/Introduction/Introduction.html#//apple_ref/doc/uid/TP40013043}{SpriteKit framework} to implement all of these effects.

\section{Who is Working on the Project}
Me! I am flying solo on this one. I will do all the things! 

\section{Hardware Requirements}
I have an iPhone to test the app and a Mac to develop on; I will not need 
any additional hardware.

\section{Project Timeline}

\begin{itemize}
  \item October 25 - Create XCode project, begin core backend 
    \begin{itemize}
      \item Set up User and Round models
      \begin{itemize}
        \item User can be a Host or a Player
        \item Allow Hosts to create a Round
        \item Allow Hosts to invite Users to a Round 
        \item Allow Hosts to assign targets to Users
      \end{itemize}
    \end{itemize}
  \item October 28 - Begin writing final project proposal
  \item October 30 - Final project proposal due, download OpenCV for iOS
  \item November 1 - Finish core backend, begin UI, weekly progress check
    \begin{itemize}
      \item Make sure everything from October 25th is working and bug-free
      \item Allow Players to recognize their targets 
      \item Use OpenCV to recognize brightly colored squares in the scene
      \begin{itemize}
        \item Implement logic to prevent Players from attacking when the
          target's square is visible
        \item Place an aura particle generator at the location of the square
      \end{itemize}
    \end{itemize}
  \item November 8 - More UI, start backend for cooler features, weekly progress check 
    \begin{itemize}
      \item Allow Players to fire energy balls at their targets 
      \item Use OpenCV to create bounding bounding boxes around the target
      \item Implement logic to determine a successful attack
      \item Remove Player from Round if they have been hit
    \end{itemize}
  \item November 15 - Catch up (buffer time in case there is a delay 
    in the schedule), weekly progress check
  \item November 22 - Begin final project write up, polish code and iron out all current bugs, 
    weekly progress check
  \item November 29 - Finish final project write up, begin bonus features (e.g.\ traps and powerups),
    weekly progress check
  \item December 6 - Final project write up due, finish bonus features, weekly progress check
  \item December 13 - Practice demo, iron out all remaining bugs
  \item December 14 - Demo
\end{itemize}

\end{document}
